\section*{Summary of the Proposal}
This project aims to uncover how different patterns of immigration effects microorganism communities. I will computationally simulate single species invasions into a network of microorganisms using either neutral theory or niche theory. To simulate the invasion dynamics I will be building on the existing microbiome model used in Pawar Lab.

\section*{Background}
Vila et al (2019) found that microbiome diversity, community size and relative fitness drives invasion success. 

\section*{Goal and Objectives}
The main focus of the project is to analyse how microbiomes respond to trickle invasions (invasive species slowly released into microbiome) compared to flood invasions (high magnitude of invasive species released simultaneously into microbiome).  \\ A possible outcome of this project is to be able to define the minimum diversity required to still successfully resist invasion.

\section*{Methods}
\subsection{Possible Variations in Invasion dynamics}
\begin{itemize}
    \item trickle or flood invasions
    \item microbiome size
    \item microbiome stability
    \item relative fitness of invasive species and microbiome species
\end{itemize}

Would a successful invasion be the incorporation of the invasive species into a stable micriobiome or just stable invasive species population?
Diversity could be used as a proxy for stability to measure the response of the micriobiomes during/after invasion. 